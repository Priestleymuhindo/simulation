% Options for packages loaded elsewhere
\PassOptionsToPackage{unicode}{hyperref}
\PassOptionsToPackage{hyphens}{url}
%
\documentclass[
]{book}
\usepackage{amsmath,amssymb}
\usepackage{lmodern}
\usepackage{iftex}
\ifPDFTeX
  \usepackage[T1]{fontenc}
  \usepackage[utf8]{inputenc}
  \usepackage{textcomp} % provide euro and other symbols
\else % if luatex or xetex
  \usepackage{unicode-math}
  \defaultfontfeatures{Scale=MatchLowercase}
  \defaultfontfeatures[\rmfamily]{Ligatures=TeX,Scale=1}
\fi
% Use upquote if available, for straight quotes in verbatim environments
\IfFileExists{upquote.sty}{\usepackage{upquote}}{}
\IfFileExists{microtype.sty}{% use microtype if available
  \usepackage[]{microtype}
  \UseMicrotypeSet[protrusion]{basicmath} % disable protrusion for tt fonts
}{}
\makeatletter
\@ifundefined{KOMAClassName}{% if non-KOMA class
  \IfFileExists{parskip.sty}{%
    \usepackage{parskip}
  }{% else
    \setlength{\parindent}{0pt}
    \setlength{\parskip}{6pt plus 2pt minus 1pt}}
}{% if KOMA class
  \KOMAoptions{parskip=half}}
\makeatother
\usepackage{xcolor}
\IfFileExists{xurl.sty}{\usepackage{xurl}}{} % add URL line breaks if available
\IfFileExists{bookmark.sty}{\usepackage{bookmark}}{\usepackage{hyperref}}
\hypersetup{
  pdftitle={Introduction à la crossvalidation},
  pdfauthor={Priestley Muhindo},
  hidelinks,
  pdfcreator={LaTeX via pandoc}}
\urlstyle{same} % disable monospaced font for URLs
\usepackage{color}
\usepackage{fancyvrb}
\newcommand{\VerbBar}{|}
\newcommand{\VERB}{\Verb[commandchars=\\\{\}]}
\DefineVerbatimEnvironment{Highlighting}{Verbatim}{commandchars=\\\{\}}
% Add ',fontsize=\small' for more characters per line
\usepackage{framed}
\definecolor{shadecolor}{RGB}{248,248,248}
\newenvironment{Shaded}{\begin{snugshade}}{\end{snugshade}}
\newcommand{\AlertTok}[1]{\textcolor[rgb]{0.94,0.16,0.16}{#1}}
\newcommand{\AnnotationTok}[1]{\textcolor[rgb]{0.56,0.35,0.01}{\textbf{\textit{#1}}}}
\newcommand{\AttributeTok}[1]{\textcolor[rgb]{0.77,0.63,0.00}{#1}}
\newcommand{\BaseNTok}[1]{\textcolor[rgb]{0.00,0.00,0.81}{#1}}
\newcommand{\BuiltInTok}[1]{#1}
\newcommand{\CharTok}[1]{\textcolor[rgb]{0.31,0.60,0.02}{#1}}
\newcommand{\CommentTok}[1]{\textcolor[rgb]{0.56,0.35,0.01}{\textit{#1}}}
\newcommand{\CommentVarTok}[1]{\textcolor[rgb]{0.56,0.35,0.01}{\textbf{\textit{#1}}}}
\newcommand{\ConstantTok}[1]{\textcolor[rgb]{0.00,0.00,0.00}{#1}}
\newcommand{\ControlFlowTok}[1]{\textcolor[rgb]{0.13,0.29,0.53}{\textbf{#1}}}
\newcommand{\DataTypeTok}[1]{\textcolor[rgb]{0.13,0.29,0.53}{#1}}
\newcommand{\DecValTok}[1]{\textcolor[rgb]{0.00,0.00,0.81}{#1}}
\newcommand{\DocumentationTok}[1]{\textcolor[rgb]{0.56,0.35,0.01}{\textbf{\textit{#1}}}}
\newcommand{\ErrorTok}[1]{\textcolor[rgb]{0.64,0.00,0.00}{\textbf{#1}}}
\newcommand{\ExtensionTok}[1]{#1}
\newcommand{\FloatTok}[1]{\textcolor[rgb]{0.00,0.00,0.81}{#1}}
\newcommand{\FunctionTok}[1]{\textcolor[rgb]{0.00,0.00,0.00}{#1}}
\newcommand{\ImportTok}[1]{#1}
\newcommand{\InformationTok}[1]{\textcolor[rgb]{0.56,0.35,0.01}{\textbf{\textit{#1}}}}
\newcommand{\KeywordTok}[1]{\textcolor[rgb]{0.13,0.29,0.53}{\textbf{#1}}}
\newcommand{\NormalTok}[1]{#1}
\newcommand{\OperatorTok}[1]{\textcolor[rgb]{0.81,0.36,0.00}{\textbf{#1}}}
\newcommand{\OtherTok}[1]{\textcolor[rgb]{0.56,0.35,0.01}{#1}}
\newcommand{\PreprocessorTok}[1]{\textcolor[rgb]{0.56,0.35,0.01}{\textit{#1}}}
\newcommand{\RegionMarkerTok}[1]{#1}
\newcommand{\SpecialCharTok}[1]{\textcolor[rgb]{0.00,0.00,0.00}{#1}}
\newcommand{\SpecialStringTok}[1]{\textcolor[rgb]{0.31,0.60,0.02}{#1}}
\newcommand{\StringTok}[1]{\textcolor[rgb]{0.31,0.60,0.02}{#1}}
\newcommand{\VariableTok}[1]{\textcolor[rgb]{0.00,0.00,0.00}{#1}}
\newcommand{\VerbatimStringTok}[1]{\textcolor[rgb]{0.31,0.60,0.02}{#1}}
\newcommand{\WarningTok}[1]{\textcolor[rgb]{0.56,0.35,0.01}{\textbf{\textit{#1}}}}
\usepackage{longtable,booktabs,array}
\usepackage{calc} % for calculating minipage widths
% Correct order of tables after \paragraph or \subparagraph
\usepackage{etoolbox}
\makeatletter
\patchcmd\longtable{\par}{\if@noskipsec\mbox{}\fi\par}{}{}
\makeatother
% Allow footnotes in longtable head/foot
\IfFileExists{footnotehyper.sty}{\usepackage{footnotehyper}}{\usepackage{footnote}}
\makesavenoteenv{longtable}
\usepackage{graphicx}
\makeatletter
\def\maxwidth{\ifdim\Gin@nat@width>\linewidth\linewidth\else\Gin@nat@width\fi}
\def\maxheight{\ifdim\Gin@nat@height>\textheight\textheight\else\Gin@nat@height\fi}
\makeatother
% Scale images if necessary, so that they will not overflow the page
% margins by default, and it is still possible to overwrite the defaults
% using explicit options in \includegraphics[width, height, ...]{}
\setkeys{Gin}{width=\maxwidth,height=\maxheight,keepaspectratio}
% Set default figure placement to htbp
\makeatletter
\def\fps@figure{htbp}
\makeatother
\setlength{\emergencystretch}{3em} % prevent overfull lines
\providecommand{\tightlist}{%
  \setlength{\itemsep}{0pt}\setlength{\parskip}{0pt}}
\setcounter{secnumdepth}{5}
\usepackage{booktabs}
\ifLuaTeX
  \usepackage{selnolig}  % disable illegal ligatures
\fi
\usepackage[]{natbib}
\bibliographystyle{apalike}

\title{Introduction à la crossvalidation}
\author{Priestley Muhindo}
\date{2021-05-31}

\begin{document}
\maketitle

{
\setcounter{tocdepth}{1}
\tableofcontents
}
\hypertarget{pruxe9requis}{%
\chapter{Prérequis}\label{pruxe9requis}}

\hypertarget{importation}{%
\chapter{Importation de Données}\label{importation}}

\hypertarget{chargement-des-libraries}{%
\section{Chargement des libraries}\label{chargement-des-libraries}}

\begin{Shaded}
\begin{Highlighting}[]
\FunctionTok{library}\NormalTok{(tidymodels)  }

\FunctionTok{library}\NormalTok{(readr)       }
\FunctionTok{library}\NormalTok{(broom.mixed) }\CommentTok{\# convertir les résultats en tibble}
\FunctionTok{library}\NormalTok{(dotwhisker) }\CommentTok{\# Visualisation de données}
\FunctionTok{library}\NormalTok{(forcats)}
\end{Highlighting}
\end{Shaded}

\hypertarget{importation-des-donnuxe9es}{%
\section{importation des données}\label{importation-des-donnuxe9es}}

Les données des oursins issues d'une expériaence au laboratoire

\begin{Shaded}
\begin{Highlighting}[]
\NormalTok{urchins }\OtherTok{\textless{}{-}}\FunctionTok{read\_csv}\NormalTok{(}\StringTok{"https://tidymodels.org/start/models/urchins.csv"}\NormalTok{) }\SpecialCharTok{\%\textgreater{}\%} \FunctionTok{setNames}\NormalTok{(}\FunctionTok{c}\NormalTok{(}\StringTok{"regime\_alim"}\NormalTok{,}\StringTok{"taille"}\NormalTok{,}\StringTok{"largeur"}\NormalTok{)) }\SpecialCharTok{\%\textgreater{}\%} 
  \FunctionTok{mutate}\NormalTok{(}\AttributeTok{regime\_alim=}\FunctionTok{factor}\NormalTok{(regime\_alim))}
\NormalTok{urchins}\SpecialCharTok{$}\NormalTok{regime\_alim }\OtherTok{\textless{}{-}} \FunctionTok{fct\_recode}\NormalTok{(urchins}\SpecialCharTok{$}\NormalTok{regime\_alim ,}
                                  \StringTok{"initial"} \OtherTok{=} \StringTok{"Initial"}\NormalTok{,}
                                  \StringTok{"Pauvre"} \OtherTok{=} \StringTok{"Low"}\NormalTok{,}
                                  \StringTok{"Riche"} \OtherTok{=} \StringTok{"High"}\NormalTok{)}
\end{Highlighting}
\end{Shaded}

\hypertarget{visualisation-des-donnuxe9es}{%
\section{Visualisation des données}\label{visualisation-des-donnuxe9es}}

\begin{Shaded}
\begin{Highlighting}[]
\FunctionTok{ggplot}\NormalTok{(urchins,}
       \FunctionTok{aes}\NormalTok{(}\AttributeTok{x=}\NormalTok{taille,}\AttributeTok{y=}\NormalTok{largeur,}\AttributeTok{group=}\NormalTok{regime\_alim,}\AttributeTok{color=}\NormalTok{regime\_alim))}\SpecialCharTok{+}
  \FunctionTok{geom\_point}\NormalTok{(}\AttributeTok{size=}\DecValTok{3}\NormalTok{)}\SpecialCharTok{+}
  \FunctionTok{geom\_smooth}\NormalTok{(}\AttributeTok{method =} \StringTok{"lm"}\NormalTok{,}\AttributeTok{se=}\NormalTok{F)}\SpecialCharTok{+}
  \FunctionTok{theme\_bw}\NormalTok{()}\SpecialCharTok{+}
  \FunctionTok{labs}\NormalTok{(}\AttributeTok{title =} \StringTok{"Regréssion de la taille des ursins }\SpecialCharTok{\textbackslash{}n}\StringTok{en fonction de la largeur de leurs largeurs }\SpecialCharTok{\textbackslash{}n}\StringTok{par régime alimentaire"}\NormalTok{)}
\end{Highlighting}
\end{Shaded}

\includegraphics{Crossvalidation_files/figure-latex/regression linéaire par regime alimentaire-1.pdf}
On peut vite remarquer que que la différence se dégage entre les différents régimes alimentaires

\hypertarget{ruxe9gression-linuxe9aire}{%
\chapter{Régression Linéaire}\label{ruxe9gression-linuxe9aire}}

Nous allons utiliser un modèle linéaire avec variables qualitatives en mettant en exergue l'interraction car nous avons à la fois un prédicteur continu et un prédicteur catégoriel. Étant donné que les pentes semblent être différentes pour au moins deux des régimes d'alimentation, construisons un modèle qui permet des interactions bidirectionnelles. Spécifier une formule R avec nos variables de cette manière :
\#\# Formule du modele et package

\begin{Shaded}
\begin{Highlighting}[]
\NormalTok{lm\_mod }\OtherTok{\textless{}{-}} 
  \FunctionTok{linear\_reg}\NormalTok{() }\SpecialCharTok{\%\textgreater{}\%} \CommentTok{\# Précision du modèle}
  \FunctionTok{set\_engine}\NormalTok{(}\StringTok{"lm"}\NormalTok{) }\CommentTok{\# Précision du package}
\end{Highlighting}
\end{Shaded}

\hypertarget{partitionnement-ruxe9echantillonage}{%
\chapter{Partitionnement (réechantillonage)}\label{partitionnement-ruxe9echantillonage}}

\hypertarget{cross-validation}{%
\chapter{Cross-validation}\label{cross-validation}}

\hypertarget{meilleurs-hyperparamuxe8tres}{%
\chapter{Meilleurs hyperparamètres}\label{meilleurs-hyperparamuxe8tres}}

  \bibliography{book.bib,packages.bib}

\end{document}
